\documentclass[a4paper,12pt,oneside]{article}%FinalView
%\documentclass[draft,a4paper,12pt,oneside]{article}%draftView

\input{settings/preamble.tex}%Inputs Preamble for its .tex
\usepackage[color]{showkeys}%Shows cross ref paths

\begin{document}
\thispagestyle{empty} % don't show (roman) page number on titlepage
\begin{titlepage}
\begin{center}
\vspace*{1cm}
	
%Title
\textbf{\LARGE{An assessment of structural enthalpy and crystallization pathways of \MgZnCa~ bulk metallic glass and amorphous films}}

\vspace{2cm}

\textbf{Scott Gleason, David Miskovic, Nicholas Hamilton, Kevin Laws, Michael Ferry}

\vspace{2cm}

UNSW Australia\\ School of Material Science and Engineering

\vspace{2cm}

%Month and Year
\today
\end{center}
\end{titlepage}

\clearpage 
\pagenumbering{roman}

%%%%%%%%%%%%%%%%%%%%%%%%%%%%%%%%%%%%%%%%%%%%%%%%%%%%%%%%%%%%%%%%%%%%%%%%%%
%\twocolumn

\section*{ABSTRACT}
\addcontentsline{toc}{section}{ABSTRACT}

The structural nature and thermal stability of amorphous alloys is highly dependent on the method by which they are produced, i.e. their relaxation rate upon cooling.  Both bulk samples and metallic glass films of \MgZnCa~ were produced by copper mold casting and \gls{dc} magnetron sputtering onto aluminium substrates, respectively. Comparisons between structural enthalpy, crystallization pathways, relaxation and crystallization kinetics of the bulk samples and films were examined by elevated temperature \acrshort{xrd} and \acrshort{dsc}. Compared with equivalent experiments on the bulk alloy, results for the thin films show distinct differences in structural enthalpy and deviations from the expected crystalline phase evolution, displaying minor peak shifts, failure of some phases to evolve, and variations in the evolution rates. 

%%%%%%%%%%%%%%%%%%%%%%%%%%%%%%%%%%%%%%%%%%%%%%%%%%%%%%%%%%%%%%%%%%%%%%%%%%

%Table of Contents
%\clearpage
\newpage
\tableofcontents\newpage
\addcontentsline{toc}{section}{TABLE OF CONTENTS}
\clearpage %% start of main matter

%%%%%%%%%%%%%%%%%%%%%%%%%%%%%%%%%%%%%%%%%%%%%%%%%%%%%%%%%%%%%%%%%%%%%%%%%%

\section{INTRODUCTION}
\pagenumbering{arabic}
\glsresetall

The structural nature and thermal stability of amorphous alloys is highly dependent on the method by which they are produced, i.e. their relaxation rate upon cooling.  Both bulk samples and metallic glass films of \MgZnCa~ were produced by copper mold casting and \gls{dc} magnetron sputtering onto aluminium substrates, respectively. Comparisons between structural enthalpy, crystallization pathways, relaxation and crystallization kinetics of the bulk samples and films were examined by elevated temperature \acrshort{xrd} and \acrshort{dsc}. Compared with equivalent experiments on the bulk alloy, results for the thin films show distinct differences in structural enthalpy and deviations from the expected crystalline phase evolution, displaying minor peak shifts, failure of some phases to evolve, and variations in the evolution rates. 

Key sources \cite{Zhang2013, Zhang2012} 
\cite{Zhang2011}

%%%%%%%%%%%%%%%%%%%%%%%%%%%%%%%%%%%%%%%%%%%%%%%%%%%%%%%%%%%%%%%%%%%%%%%%%%

\section{METHOD}

\subsection{Master alloy}
The master alloy of \MgZnCa~ was produced using high-purity elements of Mg (99.85 wt\%), Zn (99.995 wt\%), and Ca (99.8 wt\%). The alloy was prepared by induction melting in boron nitride coated graphite crucibles, purged with Ar (99.997 vol.\% purity) five times, and protected with a circulating Ar atmosphere. Alloy homogeneity was ensured by heating and cooling through a cycle of 700\degree C, 385\degree C, 650\degree C, 385\degree C, 650\degree C to a casting temperature of 500 \degree C and 450\degree C for injection and gravity casting respectively. Bulk amorphous \MgZnCa~ rods of $2.5 mm$ diameter and plates of thickness of $XX \mu m$ were produced by copper mold injection casting. The $25.4 mm$ diameter targets were prepared from a cylindrical copper mold gravity castings sectioned to thicknesses of $3.25 mm$. All samples and targets were stored under Ar when not being examined or used. 

\subsection{\acrshort{dc} magnetron sputtering}
Films were produced from an in-house \acrshort{dc} magnetron sputtering facility with Ar working gas (99.997 vol.\% purity). The power was $15W$, typical voltage of $290-350V$, nominal chamber pressure of 1 bar, substrate temperature of $25$\degree C, and Ar flow of 3.01 \acrshort{sccm}. Films were deposited directly onto to Al \acrshort{dsc} lid substrates. Depositions were for a period of 35 minutes. Deposition rate was estimated at $1.2 nm/s$. 

\subsection{Stylus profiler analysis}
Nominal film thickness was measure by a stylus profiler (Dektak 2A, Bruker, Germany). A glass slide was placed under the substrates within the sputtering chamber, allowing the substrates to act as a mask. Profile measurements were taken by measuring the height difference between the bare glass and the film coated glass. This film thickness was used to estimate the sputter deposition rate.  

\subsection{\acrshort{eds} analysis}
Alloy composition and homogeneity were confirmed by \acrshort{sem}-\acrshort{eds} (S3400, Hitachi, ?Japan?). Hyper-maps were collected with a accelerating voltage of $15-20kV$, and a probe current of $50 \mu A$. (Conditions; counts were 5000 kps or better, dead time was less than 20 \%, and working distance was 10mm). 

\subsection{\acrshort{dsc} characterization}
Isochronic \acrshort{dsc} (204 F1 Phoenix, Netzsch, Selb, Germany) was carried out in Al crucibles under a protective Ar atmosphere (99.997 vol.\% purity). Scans were performed at heating rates of $5$ to $100 K/min$. 

Isothermal relaxation \acrshort{dsc} was preformed by heating samples at $20 K/min$ to the desired annealing temperature, holding for desired time, and Ar quenching to room temperature.

For annealed \acrshort{xrd} the samples were heat treated in the \acrshort{dsc} by heating to the desired temperature at $20 K/min$ followed by Ar quenching to room temperature.

\subsection{\acrshort{xrd} characterization}
Annealing \acrshort{xrd} (Empyrean, PANalytical, Cu $K_{\alpha}$ X-ray source, $\lambda = 1.541 \angstrom$) was performed at room temperature. 
(Generator Voltage 45, Tube Current 40, Scan Step Size 0.0262606, Time per Step 397.29). 

Dynamic \acrshort{xrd} (D8, Bruker, Cu $K_{\alpha}$ X-ray source, $\lambda = 1.541 \angstrom$) was performed by raising temperature at a rate of $20 K/min$ and performing scans \textit{in situ}. The first scan was performed at $35$\degree C, then $75$\degree C, after which temperature was raised in $5K$ increments until reaching the peak temperature at $185$\degree C. The $2 \theta$ scans from $31 - 60$\degree~ were completed within $1092 sec$ ($18min,~ 12sec$) to minimise the effects of recrystallisation during the experiment. 
(Generator Voltage 45, Tube Current 100, Scan Step Size 0.02, Time per Step 134.4). 

%%%%%%%%%%%%%%%%%%%%%%%%%%%%%%%%%%%%%%%%%%%%%%%%%%%%%%%%%%%%%%%%%%%%%%%%%%

\section{RESULTS}
\subsection{Alloy composition}

From the 35 minute depositions a nominal film thickness of $2.5 \mu m$ was obtained, giving a deposition rate of approximately $1.2 nm/s$. The temperature within the chamber was found to rise $3 - 4$\degree C, significantly less than the expected $20K$ suggested by similar setups \cite{Wang2014}.

\acrshort{eds} analysis shows good agreement in the nominal composition for both the bulk and film \MgZnCa, see Table \ref{tab:EDS_Composition}.

\begin{table}[h]
	\centering
	\begin{tabular}{ c c c }
		\toprule
		\acrshort{eds} Analysis & Bulk (at\%)  & Film (at\%)  \\
		\midrule
		Mg & $64.85 \pm 3.18$ & $62.92 \pm 3.24$ \\
		Zn & $29.55 \pm 0.82$ & $31.17 \pm 0.95$ \\
		Ca & $~~ 5.60 \pm 0.17$ & $~~ 5.91 \pm 0.19$ \\ 
		\bottomrule
	\end{tabular}
	\caption{\acrshort{eds} composition of bulk and film \MgZnCa~ in atomic weight percent.}
	\label{tab:EDS_Composition}
\end{table}

\subsection{\acrshort{dsc}}
\subsubsection{Isothermal \acrshort{dsc}}
Isothermal \acrshort{dsc} was performed on the bulk and film \MgZnCa~ to examine the thermal properties. The bulk alloy was relaxed at $120$\degree C for 10 minutes before \acrshort{dsc} measurements to ensure the \Tg~ was clearly visible. The film was not relaxed as unlike the bulk the lost in free volume from relaxation would be significant and make differences between the samples much more difficult to observe [source needed???]. 

The bulk \MgZnCa~ was examined at \glspl{ht} of 5, 10, 15, 20, 30, 40, 60, 80, and 100 $K/min$ to observe changes in the \Tg~ and \Tx s with \gls{ht}. As expected greater \gls{ht} resulted in greater signal strength, exothermic peaks shifting to higher start temperatures, and an increase in thermal lag resulting in later exothermic finish temperatures and curve convolution. With this convolution the \Tg~ and \Tx $_{1}$  remained clearly visible for all \glspl{ht}, but \Tx $_{2,4,5}$ were only visible at low \glspl{ht}, and \Tx $_{3}$ was not clear at any \gls{ht}, see Figure \ref{fig:DSC_vHeatingRate_Bulk}.


START HERE. Likely talking about Films!!!
Both the bulk and film \MgZnCa~ were examined at various \acrshort{dsc} heating rates (\acrshort{ht}) to observe changed in the \Tg~ and \Tx~ temperatures. As expected higher heating rates resulted in great signal strength, exothermic peaks shifting to higher temperatures, and increases in thermal lag resulting in curve convolution. 


%single image
\begin{figure}[b]
	\centering
	\includegraphics[width=0.65\textwidth]{DSC_Fragility_Bulk.png}
	\caption[Table of contents Capition]{Bulk \MgZnCa~ relaxed at 120 \degree C for 10 minutes and heated at various heating rates. The insert stacks the \gls{dsc} curves and labels the \Tg~ and \Tx es of the $100 K/min$ sample.}%global caption
	\label{fig:DSC_vHeatingRate_Bulk}
\end{figure}

%single image
\begin{figure}[b]
	\centering
	\includegraphics[width=0.65\textwidth]{DSC_Fragility_Film.png}
	\caption[Table of contents Capition]{Unrelaxed film \MgZnCa~ heated at various heating rates. The insert stacks the \gls{dsc} curves and labels the \Tg~ and \Tx es of the $100 K/min$ sample.}%global caption
	\label{fig:DSC_vHeatingRate_Film}
\end{figure}

\subsubsection{Fragility}

Variable heating rate \acrshort{dsc} for both the bulk and film \MgZnCa~ was used to establish the fragility of the system. Numerical solutions where used to fit the equation $\beta^{-1} = \tau_{0}~ e^{(\frac{D^{*}T_{0}}{T-T_{0}})}$ \cite{Busch1998} for both the bulk and film. The fragility $m$ could then be calculated from the relationship $D^{*}=590/(m-16)$ \cite{Angell2002, Wei2014}.

For the bulk it was found $\beta^{-1} = 1.338E - 16e^{5274 (\frac{1}{T-T_{0}})}$ with Adj. $R^{2}=0.972$, giving a $D^{*}=20.4$, and a fragility $m=44.9$. For the film $\beta^{-1} = 5.921E - 11e^{2766 (\frac{1}{T-T_{0}})}$ with Adj. $R^{2}=0.861$, giving a $D^{*}=10.0$, and fragility $m=75.0$. See Figure \ref{fig:Fragility_BulkFilm_mValue}.

%single image
\begin{figure}[b]
	\centering
	\includegraphics[width=0.95\textwidth]{Bulk_Film_Fragility.png}
	\caption[Table of contents Capition]{Fitted fragility for the \MgZnCa system obtained by \acrshort{dsc} at various heating rates}
	\label{fig:Fragility_BulkFilm_mValue}
\end{figure}

\subsection{\acrshort{dsc} deconvolution}

%MultiFigure
\begin{figure}[b]
	\centering
	\includegraphics[width=.3\textwidth]{input_Raw_Bulk_NoLeveling_5Ks_result_A5lsc.png}\quad
	\includegraphics[width=.3\textwidth]{input_Raw_Bulk_NoLeveling_10Ks_result_A5lsc.png}\quad
	\includegraphics[width=.3\textwidth]{input_Raw_Bulk_NoLeveling_15Ks_result_A5lsc.png}
	\medskip
	\includegraphics[width=.3\textwidth]{input_Raw_Bulk_NoLeveling_20Ks_result_A5lsc.png}\quad
	\includegraphics[width=.3\textwidth]{input_Raw_Bulk_NoLeveling_30Ks_result_A5lsc.png}\quad
	\includegraphics[width=.3\textwidth]{input_Raw_Bulk_NoLeveling_40Ks_result_A5lsc.png}
	\medskip
	\includegraphics[width=.3\textwidth]{input_Raw_Bulk_NoLeveling_60Ks_result_A5lsc.png}\quad
	\includegraphics[width=.3\textwidth]{input_Raw_Bulk_NoLeveling_80Ks_result_A5lsc.png}\quad
	\includegraphics[width=.3\textwidth]{input_Raw_Bulk_NoLeveling_100Ks_result_A5lsc.png}
	\caption{\acrshort{dsc} deconvolution for the bulk. From left to right, top to bottom, 5, 10, 15, 20, 30, 40, 60, 80, 100 K/min.}
	\label{fig:DSC_Bulk_Decon}
\end{figure}

%MultiFigure
\begin{figure}[b]
	\centering
	\includegraphics[width=.3\textwidth]{input_Raw_Film_NoLeveling_15Ks_result_A5lsc.png}\quad
	\includegraphics[width=.3\textwidth]{input_Raw_Film_NoLeveling_20Ks_result_A5lsc.png}\quad
	\includegraphics[width=.3\textwidth]{input_Raw_Film_NoLeveling_30Ks_result_A5lsc.png}
	\medskip
	\includegraphics[width=.3\textwidth]{input_Raw_Film_NoLeveling_40Ks_result_A5lsc.png}\quad
	\includegraphics[width=.3\textwidth]{input_Raw_Film_NoLeveling_60Ks_result_A5lsc.png}\quad
	\includegraphics[width=.3\textwidth]{input_Raw_Film_NoLeveling_80Ks_result_A5lsc.png}
	\medskip
	\includegraphics[width=.3\textwidth]{input_Raw_Film_NoLeveling_100Ks_result_A5lsc.png}
	\caption{\acrshort{dsc} deconvolution for the film. From left to right, top to bottom, 15, 20, 30, 40, 60, 80, 100 K/min.}
	\label{fig:DSC_Film_Decon}
\end{figure}

\subsubsection{Onset determination}

Both the bulk and film \MgZnCa~ were examined at various \acrshort{dsc} heating rates to observe changed in the \Tg and \Tx temperatures. As expected higher heating rates resulted endothermic peaks shifting to higher temperatures and curve convolution. 

Numerical solutions were used to deconvolution the \acrshort{dsc} data so the various \Tx onsets could be determined. The results of the deconvolution are presented in Table \ref{tab:BulkOnsets} for the bulk and Table \ref{tab:FilmOnsets} for the film.

\begin{table}[h]
	\centering
	\begin{tabular}{ c c c c c c c }
		\toprule
		Heating Rate \acrshort{ht} & \acrshort{Tg} & $T_{x1}$ & $T_{x2}$ & $T_{x3}$ & $T_{x4}$ & $T_{x5}$ \\ 
		$K/min$ & & & & & & \\
		\midrule
		100 & 136.1 & 164.8 & 193.4 & 201.8 & 240.2 & 262.4 \\
		80  & 132.0 & 160.0 & 194.4 & 201.9 & 238.2 & 260.3 \\
		60  & 129.6 & 157.7 & 190.0 & 197.8 & 232.9 & 259.0 \\
		40  & 126.6 & 155.2 & 189.0 & 200.0 & 226.4 & 254.7 \\
		30  & 126.2 & 151.5 & 187.0 & 198.4 & 221.0 & 251.1 \\
		20  & 125.1 & 149.8 & 188.4 & 197.0 & 216.0 & 246.8 \\
		15  & 123.8 & 148.3 & 186.2 & 195.6 & 212.2 & 243.9 \\
		10  & 123.5 & 144.5 & 183.4 & 192.9 & 207.4 & 239.8 \\
		5   & 120.5 & 141.1 & 179.7 & 187.5 & 199.8 & 232.7 \\ 
		\bottomrule
	\end{tabular}
	\caption{Bulk \MgZnCa~ alloy onset temperatures for the various \acrshort{dsc} \acrlongpl{ht} \acrshort{ht}. All temperatures are in \degree C.}
	\label{tab:BulkOnsets}
\end{table}

\begin{table}[h]
	\centering
	\begin{tabular}{ c c c c c c c }
		\toprule
		Heating Rate \acrshort{ht} & \acrshort{Tg} & $T_{x1}$ & $T_{x2}$ & $T_{x3}$ & $T_{x4}$ & $T_{x5}$ \\ 
		$K/min$ & & & & & & \\
		\midrule
		100 & 108.5 & 128.6 &  & 177.3 &  & 240.3 \\
		80  & 106.0 & 121.2 &  & 165.6 &  & 238.8 \\
		60  & 107.3 & 134.0 &  & 176.1 &  & 237.8 \\
		40  & 100.2 & 119.8 &  & 170.7 &  & 234.2 \\
		30  & 95.3  & 110.4 &  & 169.5 &  & 232.5 \\
		20  & 95.5  & 115.2 &  & 170.5 &  & 229.4 \\
		15  & 92.5  & 113.5 &  & 168.8 &  & 224.0 \\
		\bottomrule
	\end{tabular}
	\caption{Film \MgZnCa~ alloy onset temperatures for the various \acrshort{dsc} \acrlongpl{ht} \acrshort{ht}. All temperatures are in \degree C.}
	\label{tab:FilmOnsets}
\end{table}

%single image
\begin{figure}[b]
	\centering
	\includegraphics[width=0.55\textwidth]{Bulk_Onset_Peaks_Relaxed_120C.png}
	\caption[Table of contents Capition]{The \Tg s and \Tx es of the bulk \MgZnCa~ at all heating rates. }%global caption
	\label{fig:DSC_Onsets_Bulk}
\end{figure}

%single image
\begin{figure}[b]
	\centering
	\includegraphics[width=0.55\textwidth]{Onsets_BulkandFilm.png}
	\caption[Table of contents Capition]{The \Tg s and \Tx es of the bulk and film \MgZnCa~ at all heating rates. }%global caption
	\label{fig:DSC_Onsets_BulkFilm}
\end{figure}

\subsubsection{Reaction enthalpy}

%MultiFigure
\begin{figure}[b]
	\centering
	\includegraphics[width=.6\textwidth]{Decon_Onsets_BR.png}
	\medskip
	\includegraphics[width=.6\textwidth]{Decon_peak_area_BR.png}
	\caption{\acrshort{dsc} onset temperatures and enthalpy of formation for the bulk and film.}
	\label{fig:DSC_Decon}
\end{figure}

\subsubsection{Relaxation enthalpy}

\subsection{\acrshort{xrd}}
\subsubsection{Annealing \acrshort{xrd}}

%single image
\begin{figure}[b]
	\centering
	\includegraphics[width=0.9\textwidth]{XRD_Annealing_Bulk.png}
	\caption[Table of contents Capition]{\acrshort{xrd} pattern for Bulk \MgZnCa~ heated through several crystallization peaks identified from \acrshort{dsc}}
	\label{fig:XRD_Annealing_Bulk}
\end{figure}

%single image
\begin{figure}[b]
	\centering
	\includegraphics[width=0.9\textwidth]{XRD_Annealing_Film.png}
	\caption[Table of contents Capition]{\acrshort{xrd} pattern for Film \MgZnCa~ heated through several crystallization peaks identified from \acrshort{dsc}}
	\label{fig:XRD_Annealing_Film}
\end{figure}

\subsubsection{Dynamic \acrshort{xrd}}

%code to put 2 images side by side in a figure
\begin{figure}[b]
	\centering
	%Image 1
	\begin{subfigure}[htbp]{0.75\textwidth}
		\includegraphics[width=\textwidth]{XRD_Dynamic_Bulk.png}
		\caption{}
		\label{fig:XRD_Dynamic_FullStack_Bulk}
	\end{subfigure}
	%Image 2
	\begin{subfigure}[htbp]{0.75\textwidth}
		\includegraphics[width=\textwidth]{Bulk_Heated_XRD_Waterfall3D_Smooth2.png}
		\caption{}
		\label{fig:XRD_Dynamic_WaterFall_Bulk}
	\end{subfigure}
	\caption{(a) Stacked \gls{xrd} patterns from the incremental heating of bulk \MgZnCa. (b) Cascading \gls{xrd} patterns from the incremental heating of bulk \MgZnCa. }%global caption
	\label{fig:XRD_Dynamic_Bulk}
\end{figure}

%code to put 2 images side by side in a figure
\begin{figure}[b]
	\centering
	%Image 1
	\begin{subfigure}[htbp]{0.75\textwidth}
		\includegraphics[width=\textwidth]{XRD_Dynamic_Film.png}
		\caption{}
		\label{fig:XRD_Dynamic_FullStack_Film}
	\end{subfigure}
	%Image 2
	\begin{subfigure}[htbp]{0.75\textwidth}
		\includegraphics[width=\textwidth]{TF_Facet_HeatXRD_Waterfall3D_Smooth.png}
		\caption{}
		\label{fig:XRD_Dynamic_WaterFall_Film}
	\end{subfigure}
	\caption{(a) Stacked \gls{xrd} patterns from the incremental heating of film \MgZnCa. (b) Cascading \gls{xrd} patterns from the incremental heating of film \MgZnCa. }%global caption
	\label{fig:XRD_Dynamic_Film}
\end{figure}

%%%%%%%%%%%%%%%%%%%%%%%%%%%%%%%%%%%%%%%%%%%%%%%%%%%%%%%%%%%%%%%%%%%%%%%%%%

\section{DISCUSSION}

The use of a 60K \gls{dsc} heating rate compared to the more commonly used 20K rate [sources] shifts peaks for the bulk \MgZnCa~ alloy about 8 - 15 degrees higher. This higher heating rates were used because crystallization events for the films were difficult to differentiation at the lower heating rate. 
Films show little shift to high temperature peaks with increases heating rates, but large shifts with relaxation. 
Bulk show the opposite behaviour, larger peaks shifts with higher heating rates and little shift with relaxation.

%%%%%%%%%%%%%%%%%%%%%%%%%%%%%%%%%%%%%%%%%%%%%%%%%%%%%%%%%%%%%%%%%%%%%%%%%%

\section{CONCLUSIONS}

%%%%%%%%%%%%%%%%%%%%%%%%%%%%%%%%%%%%%%%%%%%%%%%%%%%%%%%%%%%%%%%%%%%%%%%%%%

\section{ACKNOWLEDGEMENTS}

%People
Yu Wang for his assistance with \acrshort{xrd} experimentation and Rietveld refinement. 

%%%%%%%%%%%%%%%%%%%%%%%%%%%%%%%%%%%%%%%%%%%%%%%%%%%%%%%%%%%%%%%%%%%%%%%%%%

%Bibliography
\bibliography{ThesisBib}
\bibliographystyle{unsrt}

%%%%%%%%%%%%%%%%%%%%%%%%%%%%%%%%%%%%%%%%%%%%%%%%%%%%%%%%%%%%%%%%%%%%%%%%%%


\end{document}